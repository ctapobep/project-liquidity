\section{Who needs to read this book?}
This book is mostly for software engineers, team leads and project managers, but can be useful for anyone who treats efficiency as an important factor in their work.

\section{The mission of the book}
How do you measure project quality? Do you use code coverage statistics? Do you measure documentation coverage? This book introduces a new way of measuring project quality - liquidity. The term was created as an association to liquidity of the job market and personnel of your company. The faster you can get new people in, the less expensive project is. And that's exactly the best metric that shows how high project quality is. This is similar to counting the amount of WTF in source code, the book widers this metric not only to source code, but to architecture, process, documentation, etc. We're going to go through different aspects of project to figure out how building of how-quality projects differs from building usual ones.

Working in different teams and countries I found a common thing about software engineers - they don't give a toss. About what? Hm.. about anything but current state of affairs. How often have you inherited a project from other teams which was in a terrible non-maintainable state? If it comes to me, the answer is - fairly often. Why do you think this happens? Every developer is a cool guy in his blog, but when it comes the practice, you get a mess after them. The mission of this book is to show another universe where everyone is happy about his project and people love to maintain! 

Another question: how much time usually it takes for new people that just join the project to become a full-functional unit of work that can work on tasks independently? I've seen projects where people were working for half a year and still were not-that-confident in the project, they still didn't understand how it functions and what business it does. Another mission of the book is to show how to write a project that accepts the liquidity of job market so that people can join it as fast as this is possible. How fast is that? Let's decrease 6 months to 2 weeks - that's our aim. And here were are considering complex projects connected to other systems and solving complex business tasks e.g. in areas like finance, investment, science, etc.

Why is it important for project to be liquid? As we already mentioned - it's all about money. Most of projects you participate are probably commercial and moreover - they have some budget. As a guy who works on this project, your objective is: 
\begin{enumerate}
 \item To achieve good results within small amount of time
 \item Leave the project in a maintainable state after 12 months of active development
 \item In addition to previous item - make people after you happy
 \item Still have personal life after that 
\end{enumerate}

In order to let that happen, we're going to cover several topics:
\begin{enumerate}
 \item Agility and Continuous delivery. How automation helps us?
 \item Local deployment. Why is it important to be able to run app locally?
 \item Project architecture. What aspects of architecture should we care about to make project liquid? How to choose between technologies?
 \item OOD principles. How do we follow them correctly without complicating the system and code too much?
 \item Administration means. How good administration facilities of your software can help introducing new people to the project?
 \item Documentation on different levels including code docs, project specification, environment configuration, different kind of diagrams.
 \item Logging. How logging can help you in understanding of the project?
 \item Functional tests. How to make tests run the project specification?
 \item Corporate policies and standards - where is the wise line?  
 \item Bugs, how do they impact liquidity?    
\end{enumerate}

