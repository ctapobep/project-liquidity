\begin{document}
\section{The mission of the book}

Working in different teams and countries I found a common thing about software engineers - they don't give a toss. About what? Hm.. about anything but current state of affairs. How often have you inherited a project from other teams which was in a terrible non-maintainable state? If it comes to me, the answer is - fairly often. Why do you think this happens? Every developer is a cool guy in his blog, but when it comes the practice, you get a mess after them. The mission of this book is to show another universe where everyone is happy about his project and people love to maintain! 
Another question: how much time usually it takes for new people that just join the project to become a full-functional unit of work that can work on tasks independently? I've seen projects where people were working for half a year and still were not-that-confident in the project, they still didn't understand how it functions and what business it does. Another mission of the book is to show how to write a project that accepts the liquidity of job market so that people can join it as fast as this is possible. How fast is that? Let's decrease 6 months to 2 weeks - that's our aim.
You probably are curious what does Project Liquidity mean? It shows the quality of the project, particularly how fast a new joiner can become a useful team member. The term was created as an association to liquidity of the job market and personnel of your company. 
Why is it important for project to be liquid?
\end{document}
