\documentclass[11pt,a4paper,oneside]{article}
\begin{document}
\section{Local Deployment}
Imagine yourself joining the project, what's the first thing you'd like to have? Right - the project up and running. It's crucial for people to see the software in action to understand it, and it's not enough to see it in production or on some shared environment. Any other environment is more restrictive than local one: you don't have permissions to see everything like admin panel, or you can damage the environment by pressing a red button - and you don't want to damange anything on the first day of the project, don't you?

Why this is so important? Remember our aim is to get people fully started in 2 weeks - and that's what impossible if you can't run the project easily. 

First of all because you need to understand the infrastructure: what databases are used, what protocols are there to communicate between different systems or different nodes. 

Second, you need to understand the workflow: different kind of requests can be processed differently, and this also may be dependant on current state. In order to understand the workflow developer has to debug the system - this is the only way he can feel it. Even if you have some special frameworks to build your flows like Apache Camel, Spring Integration or even Disruptor which can help you with creating and following routes of the workflow, they still can be pretty combersome and complicated. I've seen systems which were using such frameworks, but yet they were that big and splitted for the sake of DRY (don't repeat yourself principle), that it took efforts of several people who spent days to read just couple of routes. And one of the reasons - we couldn't run that system locally. So we had to guess. That's not a situation you want to appear when it comes to new stuff. You can guess and be creative when you understand things already, but not on the first days on new position.

Third thing - business domain. It's important for developer to understand the area he works with and how this or that action changes that domain. If you have a database and you're free to do any business action, you can see the results right away. If you know the impact of clicking on that button, you can figure out what that freaking term means in scope of data changes. Plus you can debug it and thus you have some means already to understand that business you're going to work with. This might be harder in a very asynchroneous architecture because it might be nearly impossible to debug concurrent code, but you still have more tools than people who can't afford local environment.

That's why you need the app started locally. Now we have a clear understanding of our first objective, let's get acquainted with problems that may stop us getting a perfect project.

There are probably some people thinking that it's not always possible, and these people might be right - not all the projects are built the way they can be started on development machine. Some software might require 16G of RAM to get started with 8 cores, and I don't think that all development machines in your company have that horsepower. 

Another case is - some projects are built the way it shouldn't be possible to run it on a single node. Some projects are built solely for distributed usage, there is no logical use case when you'd need to run it on single miserable machine.

So we have an obsticle - big and solely distributed projects can't get started easily on any environment, but what you're usually missing out - we're building this software and we're the ones who create that architecture. 
\end{document}
